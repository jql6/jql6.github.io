\documentclass[12pt]{amsart}
% 80 Character width guide
%%%%%%%%%%%%%%%%%%%%%%%%%%%%%%%%%%%%%%%%%%%%%%%%%%%%%%%%%%%%%%%%%%%%%%%%%%%%%%%%

%%% Packages %%%
\usepackage{amsfonts, amsmath, amssymb, amsthm, enumitem, geometry, graphicx}
\usepackage{hyperref, latexsym, multirow, xcolor} \usepackage[T1]{fontenc}
%\usepackage{pgfplots} \pgfplotsset{width = 8cm, compat = 1.9}
%%% Document settings %%%
\geometry{letterpaper, lmargin = 15mm, rmargin = 15mm}
\definecolor{l_gray1}{RGB}{215, 215, 215} \color{l_gray1} % Text color
\definecolor{d_gray1}{RGB}{45, 45, 45} \pagecolor{d_gray1} % Page color

\begin{document}
\title[Short Title]{Example1}
\author{J} \maketitle

Math environment:
$a = 0;\ \sum_{i = 0}^n n$\\

Similar to math environment, but it's centered and larger. If you want these
equations to be numbered, you can use the equation environment.
$$a_j^l = \sigma \left( \sum_k w_{j k}^l a_k^{l - 1} + b_j^l \right)$$

Tabular environment:\\
\begin{tabular}{| c | c |}
\hline
5 & 8\\
\hline
6 & 3\\
\hline
8 & 9\\
\hline
\end{tabular}

Array environment:\\
$$
\begin{array}{c | c}
\sum_{i = 1}^\infty a_i^n & \text{text}\\
\end{array}
$$

\begin{align*}
\delta_j^l &= \frac{\delta C}{\delta z_j^l}\\
&= \sum_k \frac{\delta C}{\delta z_k^{l + 1}} \frac{\delta z_k^{l + 1}}{\delta z_j^L}\\
&= \sum_k \frac{\delta z_k^{l + 1}}{\delta z_j^l} \delta_k^{l + 1}\\
\end{align*}

\end{document}
